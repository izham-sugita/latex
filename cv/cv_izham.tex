%%%%%%%%%%%%%%%%%%%%%%%%%%%%%%%%%%%%%%%%%
% Medium Length Professional CV
% LaTeX Template
% Version 2.0 (8/5/13)
%
% This template has been downloaded from:
% http://www.LaTeXTemplates.com
%
% Original author:
% Trey Hunner (http://www.treyhunner.com/)
%
% Important note:
% This template requires the resume.cls file to be in the same directory as the
% .tex file. The resume.cls file provides the resume style used for structuring the
% document.
%
%%%%%%%%%%%%%%%%%%%%%%%%%%%%%%%%%%%%%%%%%

%----------------------------------------------------------------------------------------
%	PACKAGES AND OTHER DOCUMENT CONFIGURATIONS
%----------------------------------------------------------------------------------------

\documentclass{resume} % Use the custom resume.cls style

\usepackage[left=0.75in,top=0.6in,right=0.75in,bottom=0.6in]{geometry} % Document margins
\usepackage{enumerate}

\name{Muhammad Izham bin Ismail} % Your name
\address{School of Mechatronic Engineering, Kampus Tetap Pauh Putra}
\address{Universiti Malaysia Perlis} % Your secondary addess (optional)
\address{(+6019)-359-5091 \\ izham@unimap.edu.my} % Your phone number and email


\begin{document}

%----------------------------------------------------------------------------------------
%	EDUCATION SECTION
%----------------------------------------------------------------------------------------

\begin{rSection}{Education}

{\bf Kyoto Institute of Technology, Kyoto, Japan} \hfill {\em Apr 2009 - March 2015} \\ 
PhD in Mechanical Engineering \\
Computational Fluid Dynamics \\
Research in lattice Boltzmann method (LBM) 

{\bf Kyoto Institute of Technology, Kyoto, Japan} \hfill {\em Oct 2005 - Mar 2007} \\ 
M. Eng. in Mechanical and System Engineering \\
Computational Fluid Dynamics \\
Research in finite difference based lattice Boltzmann method (FDLBM)
%Minor in Linguistics \smallskip \\ % small skip in line before details
%Member of Eta Kappa Nu \\
%Member of Upsilon Pi Epsilon \\
%Overall GPA: 5.678

{\bf Kyoto Institute of Technology, Kyoto, Japan} \hfill {\em Apr 1999 - Mar 2003} \\ 
B. Eng. in Mechanical and System Engineering \\
Equivalent Cummulative GPA: 3.2

\end{rSection}

%----------------------------------------------------------------------------------------
%	WORK EXPERIENCE SECTION
%----------------------------------------------------------------------------------------

\begin{rSection}{Experience}
\begin{rSubsection}{Universiti Malaysia Perlis}{Apr 2015 - Present}{Senior Lecturer, DS52}{Kampus Tetap Pauh Putra, Arau, Perlis}
\item Teaching ENT342 Computational Fluid Dynamics
\item Person-in-charge of EIT302 Industrial Training for Mechanical Engineering Program at PPK Mekatronik, UniMAP.
\end{rSubsection}

\begin{rSubsection}{Universiti Malaysia Perlis}{Mar 2007 - Mar 2015}{Lecturer, DS45}{Kampus Tetap Pauh Putra, Arau, Perlis}
\item Teaching ENT342 Computational Fluid Dynamics, ENT241 Fluid Mechanics 1 and ENT482 Mechanical Design Project 1.
\item Person-in-charge of EIT302 Industrial Training for Mechanical Engineering Program at PPK Mekatronik, UniMAP.
\item Previously taught ENT 352 Computer Aided Engineering Design, ENT 361 Digital Electronics and Applications, EKT 120 Computer Programming and DNT 123 Computer Aided Drafting.
\end{rSubsection}

\begin{rSubsection}{SONY EMCS(M) Sdn Bhd}{Jun 2005 - Aug 2005}{Mechanical Design Engineer}{Seberang Perai, Penang}
\item Conduct design for mechanical parts, specifically headphone and related peripheral devices.
\item Lead liaison with Japanese and Korean counterparts for new product introduction.
\end{rSubsection}

\begin{rSubsection}{TOWA(M) Sdn Bhd}{Apr 2005 - May 2005}{Mechanical Design Engineer}{Bayan Lepas FTZ, Penang}
\item Mechanical design for plastic injection jig.
\item Machine control troubleshooting.
\end{rSubsection}

\begin{rSubsection}{Renesas Technology(M) Sdn Bhd}{Apr 2004 - Dec 2004}{Production Engineer}{Bayan Lepas FTZ, Penang}
\item Transistor and diode line assembly maintainence and troubleshooting.
\item New product introduction team member.
\end{rSubsection}

%------------------------------------------------

%------------------------------------------------
\end{rSection}

%----------------------------------------------------------------------------------------
%	TECHNICAL STRENGTHS SECTION
%----------------------------------------------------------------------------------------
%\newpage
\begin{rSection}{Technical Strengths}

\begin{tabular}{ @{} >{\bfseries}l @{\hspace{6ex}} l }
Languages & Malay, English, Japanese \\
Programming Languages & Fortran77/90/03, C/C++, Python, Scilab, Octave and GCC compilers. \\
CAE softwares & ANSYS(Fluent), Solidworks and AutoCAD. \\
Productivity applications & LaTex, Microsoft Office and Libre Office.\\
Technical expertise & Computational fluid dynamics, numerical method,\\ 
                    & fluid dynamics, GPU and OpenMP programming, \\
& OpenCL, NVIDIA CUDA, OpenBLAS library, \\
& GNU Scientific Library and  MPICH. \\
Extra link  & https://github.com/izham-sugita
\end{tabular}

\end{rSection}


%----------------------
\begin{rSection}{Publications}
\begin{rSubsectionPub}{Journals}{}{}{}

\item \textbf{M. Izham}, T. Fukui and K. Morinishi, Simulation of three-dimensional homogeneous isotropic turbulence using the moment-based lattice Boltzmann method and LES-lattice Boltzmann method, Journal of Fluid Science and Technology (JSFT), Vol. 9 (2014), pp 1-13 
\item \textbf{M. Izham}, T. Fukui and K. Morinishi, Application of regularized lattice Boltzmann method for incompressible flow simulation at high Reynolds number and flow with curved boundary, Journal of Fluid Science and Technology (JSFT), Vol. 6 (2011), pp 812-822 

\end{rSubsectionPub}

\begin{rSubsectionPub}{Proceedings}{}{}{}

\item M N Rahman Y, Z M Razlan, \textbf{M Izham}, M I Omar, N A A Zambri, A B Shahriman, I Zunaidi and W K Wan, A Study on Possibility of CFD Simulation on Air Simulation in Minor Operation Theatre, IOP Conference Series: Materials Science and Engineering, Volume 429, conference 1, ICAMIA 2018, 5-17 August 2018 in Kuching, Sarawak, Malaysia.

\item M N Rahman Y, Z M Razlan, \textbf{M Izham}, M I Omar, N A A Zambri, A B Shahriman, I Zunaidi and W K Wan, A Test of Possibility on Relative Humidity Function in Minor Operation Theatre, IOP Conference Series: Materials Science and Engineering, Volume 429, conference 1, ICAMIA 2018, 5-17 August 2018 in Kuching, Sarawak, Malaysia.

\item \textbf{M. Izham} and M. S. Mohamad, Application of entropically damped artificial compressibility method for incompressible unsteady flow simulations, Malaysian Technical Universities Conference on Engineering and Technologies (MUCET) 2013, 3-4 Dec 2013, Universiti Malaysia Pahang.

\item M. S. Mohamad and \textbf{M. Izham}, Lattice kinetic scheme with virtual flux method for incompressible flow with complex geometry simulation,Seminar Kebangsaan Aplikasi Sains dan Matematik 2013,29-30 Oct 2013, Universiti Tun Hussein Onn Malaysia.

\item \textbf{M. Izham}, T. Fukui and K. Morinishi, A comparative study of regularized lattice Boltzmann method and entropic lattice Boltzmann method for high Reynolds number flow, Proceedings of Asian Symposium on Computational Heat Transfer and Fluid Flow 2011 (ASCHT11), Kyoto University, 22-26 Sep 2011, paper no. 091

\item \textbf{M. Izham}, T. Fukui and K. Morinishi, Regularized lattice Boltzmann method with virtual flux method for incompressible flow simulations, $24^{th}$ Computational Fluid Mechanics Symposium, Keio University, 20-22 Dec 2010, E6-4

\end{rSubsectionPub}

\end{rSection}


\begin{rSection}{Grants/Awards}
\begin{rSubsectionPub}{Grants}{}{}{}

\item A new hyperparameter estimation of deep learning to detec multi class visual field defect for optic pathway disease diagnostic, 
\textbf{Project Member}, Ministry of Education Fundamental Research Grant Scheme (FRGS), on going 1 Jan 2019 - 31 December 2020

\item Analysis and modeling of incompressible turbulent flow by implicit large-eddy simulations using moment-based lattice Boltzmann method with parallel computing implementation using graphics processing unit (GPU), \textbf{Project Leader}, Ministry of Education Fundamental Research Grant Scheme (FRGS), June 2014 - Mei 2016

\item Synergistic analysis and CFD modelling of sustainable nano porous clay-synthesized calcium carbonate for hydrogen sulphide gas adsorbent applications, 
\textbf{Project Member}, Ministry of Education Fundamental Research Grant Scheme (FRGS), June 2014 - Mei 2016

\item Characterization of Overlapping Stress Shielding and Stress Amplification Effect on Fracture Behavior of Interacting Macro-Microcracks, 
\textbf{Project Member}, Ministry of Education Fundamental Research Grant Scheme (FRGS), 23 December 2013 - 22 December 2016



\end{rSubsectionPub}
\end{rSection}


%----------------------------------------------------------------------------------------

%----------------------------------------------------------------------------------------
%	EXAMPLE SECTION
%----------------------------------------------------------------------------------------

%\begin{rSection}{Section Name}

%Section content\ldots

%\end{rSection}

%----------------------------------------------------------------------------------------


\end{document}
